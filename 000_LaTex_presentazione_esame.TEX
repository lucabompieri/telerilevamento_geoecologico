\documentclass{beamer}
\usepackage{listings}
\usepackage{color}
\usepackage[T1]{fontenc}


\usetheme{Frankfurt}
\usecolortheme{dove}

\title{Esame Telerilevamento Geoecologico}
\institute{Alma Mater Studiorum - Università di Bologna\\Telerilevamento Geo-Ecologico}
\author{Studente: Luca Bompieri\\Docente: Duccio Rocchini}
\date{A.A. 2021/2022}
\logo{\includegraphics[height=1cm]{log.png}}


\begin{document}


\maketitle

\AtBeginSection[]
{
\begin{frame}
\frametitle{Indice}
\tableofcontents[currentsection, currentsubsection]
\end{frame}
}
\AtBeginSubsection[] {
    \begin{frame}
    \frametitle{Indice} 
    \tableofcontents[currentsubsection]  
    \end{frame}
}


\section{Scopo del progetto}

\begin{frame}{Scopo del progetto}
\begin{itemize}
    \item  Mostrare la perdita di manto nevoso del ghiacciao Okjökull e delle aree limitrofe, in un intervallo di tempo di 33 anni (1986-2019).
\end{itemize}
\end{frame}

\section{Inquadramento}

\begin{frame}{Inquadramento geografico}
\begin{itemize}
    \item L'area di studio, riquadro rosso in carta, risulta essere localizzata in Inslanda, a nord-est di Reykjavík. 
\end{itemize}
    \includegraphics[width=1\textwidth]{0_inquadramento.png}
    \centering
\end{frame}

\begin{frame}{Ingrandimento area di studio}
\begin{itemize}
    \item Il riquadro rosso in carta indica l'effettiva area del  ghiacciao Okjökull. 
\end{itemize}
    \includegraphics[width=175, height=175]{ingrandimento.jpg}
    \centering
\end{frame}

\section{Metodi di analisi}

\begin{frame}{Metodi}
\begin{itemize}
    \item Caricamento e preparazione iniziale delle immagini
    
    \bigskip
    
    \item \pause Calcolo pca (principal component analysis) delle due immagini
    
    \bigskip
    
    \item \pause Calcolo della perdita di manto nevoso
    
    \bigskip
    
    \item \pause Calcolo della deviazione standard per mostrare i punti più significativi
\end{itemize}    
\end{frame}

\section{Analisi immagini}

\begin{frame}{Script usato}
    \begin{tiny}
        \lstinputlisting[language=R]{preparazione_immagini.R}
    \end{tiny}
\end{frame}

\begin{frame}{Componenti Okjökull 1986}
    \includegraphics[width=7cm, height=7cm]{okjokull_1986_components.jpg}
    \centering
\end{frame}

\begin{frame}{Componenti Okjökull 2019}
    \includegraphics[width=7cm, height=7cm]{okjokull_2019_components.jpg}
    \centering
\end{frame}

\begin{frame}{Immagini RGB comparate [1986-2019]}
    \includegraphics[width=1\textwidth]{rgb_picture.jpg}
    \centering
\end{frame}

\section{PCA}

\begin{frame}{Script usato}
    \begin{tiny}
        \lstinputlisting[language=R]{pca.R}
    \end{tiny}
\end{frame}

\begin{frame}{Componenti pca 1986}
    \includegraphics[width=7cm, height=7cm]{1986 pca.jpg}
    \centering
\end{frame}

\begin{frame}{Componenti pca 2019}
    \includegraphics[width=7cm, height=7cm]{2019 pca.jpg}
    \centering
\end{frame}

\section{Diff. neve}

\begin{frame}{Script usato}
    \begin{tiny}
        \lstinputlisting[language=R]{neve.R}
    \end{tiny}
\end{frame}

\begin{frame}{Componente 1 pca 1986}
    \includegraphics[width=7cm, height=7cm]{1986_component_1.jpg}
    \centering
\end{frame}

\begin{frame}{Componente 1 pca 2019}
    \includegraphics[width=7cm,height=7cm]{2019_component_1.jpg}
    \centering
\end{frame}

\begin{frame}{Differenza tra le due immagini}
    \includegraphics[width=7cm,height=7cm]{diff.jpg}
    \centering
\end{frame}

\section{Dev.St.}

\begin{frame}{Script usato}
    \begin{tiny}
        \lstinputlisting[language=R]{devst.R}
    \end{tiny}
\end{frame}

\begin{frame}{Mappa analisi deviazione standard}
    \includegraphics[width=7cm, height=7cm]{standard deviation map.jpg}
    \centering
\end{frame}

\section{Conclusioni}

\begin{frame}
\begin{itemize}
    \item Le immagini prodotte mostrano come,  nell'intervallo trascorso tra i due anni di studio, 1986-2019, la perdita di manto nevoso e lo scioglimento parziale dei ghiacciai siano più concentrati nelle aree basali delle strutture, mentre le aree più centrali risultano essere praticamente invariate.
    Ghiacciai di piccole dimensioni come, l'Okjökull, risultano ormai essere estinti.
\end{itemize}
\end{frame}

\begin{frame}
\begin{itemize}
    \item Grazie dell'attenzione.
\end{itemize}
\end{frame}

\end{document}

